\documentclass{article}
\usepackage[margin=0.75cm]{geometry}
\usepackage{hyperref}
\usepackage{graphicx}
\usepackage{float}
\usepackage{caption}
\usepackage[lofdepth,lotdepth]{subfig}

\title{Library for Embedded Datastream Algorithm}
\author{Martin}
\begin{document}
\maketitle
\section{Summary}
Given the increasing number of streaming devices, it has become interesting to offload 
stream mining technique onto these devices rather than waiting for the stream to reach the cloud.
OrpailleCC provide a collection of data stream algorithms developed to be deploy on embeded device.

The repository includes a wide variaty of algorithms ranging from compression \cite{ltc} to machine learning \cite{mc-nn} algorithms.
A more detailled list is available in the repository.

%*** Embracing the hack
%In the long term we should move away from closed/proprietary solutions such as Scopus and Web of Science that primarily track papers and their citations, and instead move to tools that can track things without DOIs such as //depsy.org. However, that's the long-term fix, and not the one that helps research software engineers and researchers who are already spending significant amounts of time writing code today.

%If software papers are currently the best solution for gaining career credit for software, then shouldn't we make it as easy as possible to create a software paper? Building high quality software is already a lot of work — what if we could make the process of writing a software paper take less than an hour?

%*** The Journal of Open Source Software
%The Journal of Open Source Software is an open source[3] developer-friendly journal for research software packages. It's designed to make it as easy as possible to create a software paper for your work. If a piece of software is already well documented, then paper preparation (and submission) should take no more than an hour.

%The JOSS "paper" is deliberately extremely short and only allowed to include:

%A short abstract describing the high-level functionality of the software (and perhaps a figure)
%A list of the authors of the software (together with their affiliations)
%A list of key references including a link to the software archive
%Paper are not allowed to include other things such as descriptions of API functionality, as this should be included in the software documentation. You can see an example of a paper here.

%Oh cool. You're going to publish a bunch of crappy papers!
%Not at all. Remember, software papers are just advertising and JOSS "papers" are essentially just abstracts that point to a software repository. The primary purpose of a JOSS paper is to enable citation credit to be given to authors of research software.

%We're also not going to let just any old software through: JOSS has a rigorous peer review process and a first-class editorial board highly experienced at building (and reviewing) high-quality research software.
\bibliographystyle{plain}
\bibliography{paper}
\end{document}
